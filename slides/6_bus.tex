\section[I BUS]{I BUS}
\label{sec:bus}
\sectionframe{images/covers/cover_bus.jpg}{I BUS}	 


%\subsection[Central Processing Unit]{Central Processing Unit}
%\begin{frame}
%	\frametitle{Central Processing Unit}
%	
%%	\begin{block}{Central Processing Unit}
%		Una CPU, \textbf{central processing unit} (unità centrale di elaborazione o processore centrale), indica un'unità o sottosistema logico e fisico che sovraintende alle \textbf{funzionalità logiche di elaborazione} principali di un computer.
%		La CPU è un'elaborata combinazione di transistor che può essere definita \textit{circuito integrato}.\\~\\
%		\pause
%		All'interno della CPU individuiamo tre elementi fondamentali:
%		\begin{itemize}
%			\item \textbf{la CU}, \textit{Control Unit} (l’unità di controllo):\\
%			coordina l'esecuzione delle operazioni da parte del processore;
%			\item \textbf{la ALU}, \textit{Arithmetic-Logic Unit} (l’Unità Aritmetico-Logica):\\
%			si occupa di eseguire le operazioni aritmetico-logiche;
%			\item \textbf{i registri di memoria}:\\
%			diverse \textit{celle di memoria} dedicate a scopi specifici che vengono utilizzati per il controllo dell'esecuzione di un programma.
%		\end{itemize}
%%	\end{block}
%	
%\end{frame}



\subsection[Lo scopo dei BUS]{Lo scopo dei BUS}
\begin{frame}
	\frametitle{Lo scopo dei BUS}
	
	\begin{block}{Lo scopo dei BUS}
		Un BUS nei computer è un sistema di comunicazione che trasmette dati e segnali tra le diverse componenti del sistema, come processori, memoria e dispositivi. I BUS si possono dividere in due categorie:
		\begin{itemize}
			\item Il \textbf{BUS interno} connette componenti all'interno della scheda madre, facilitando lo scambio di dati tra CPU, RAM e altre parti.
			\item Il \textbf{BUS esterno}, come il BUS di sistema, collega la scheda madre a dispositivi esterni come unità di archiviazione e schede di espansione. 
		\end{itemize}
		
		Il BUS interno è più veloce e gestisce comunicazioni tra parti vitali del computer, mentre il BUS esterno connette il computer a periferiche. Entrambi sono cruciali per l'efficienza e la coesione del sistema, consentendo il corretto funzionamento e la comunicazione tra le diverse componenti.
		
	\end{block}
	
\end{frame}
