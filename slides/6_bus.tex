\section[I BUS]{I BUS}
\label{sec:bus}
\sectionframe{images/covers/cover_bus.jpg}{I BUS}	 


%\subsection[Central Processing Unit]{Central Processing Unit}
%\begin{frame}
%	\frametitle{Central Processing Unit}
%	
%%	\begin{block}{Central Processing Unit}
%		Una CPU, \textbf{central processing unit} (unità centrale di elaborazione o processore centrale), indica un'unità o sottosistema logico e fisico che sovraintende alle \textbf{funzionalità logiche di elaborazione} principali di un computer.
%		La CPU è un'elaborata combinazione di transistor che può essere definita \textit{circuito integrato}.\\~\\
%		\pause
%		All'interno della CPU individuiamo tre elementi fondamentali:
%		\begin{itemize}
%			\item \textbf{la CU}, \textit{Control Unit} (l’unità di controllo):\\
%			coordina l'esecuzione delle operazioni da parte del processore;
%			\item \textbf{la ALU}, \textit{Arithmetic-Logic Unit} (l’Unità Aritmetico-Logica):\\
%			si occupa di eseguire le operazioni aritmetico-logiche;
%			\item \textbf{i registri di memoria}:\\
%			diverse \textit{celle di memoria} dedicate a scopi specifici che vengono utilizzati per il controllo dell'esecuzione di un programma.
%		\end{itemize}
%%	\end{block}
%	
%\end{frame}



\subsection[Lo scopo dei BUS]{Lo scopo dei BUS}


%\begin{frame}
%	\frametitle{Lo scopo dei BUS}
%	
%	\begin{block}{I BUS: le autostrade per i bit}
%		Il BUS nei computer è un sistema di comunicazione che trasmette dati e segnali tra le diverse componenti del sistema, come processori, memoria e dispositivi. Distinguiamo due categorie di BUS:
%		\begin{itemize}
%			\item L'\textbf{internal BUS} (o \textbf{system BUS}): connette componenti all'interno della scheda madre, facilitando lo scambio di dati tra CPU, RAM e altre parti.
%			\item L' \textbf{external BUS}: collega la scheda madre a dispositivi esterni come unità di archiviazione e schede di espansione. 
%		\end{itemize}
%		
%	\end{block}
%	
%\end{frame}

\begin{frame}
	\frametitle{Lo scopo dei BUS}
	
	\begin{block}{}
		I processori operano su $1^i$ e $0^i$. Gli $1^i$ e gli $0^i$ viaggiano da un punto all'altro all'interno del processore, così come all'esterno verso altri chip.
		Per spostare gli $1^i$ e gli $0^i$, vengono utilizzate \textbf{linee elettroniche} chiamate \textbf{BUS}; una sorta di autostrade per i bit. Distinguiamo tra:
		\begin{itemize}
			\item \textbf{internal BUS}: le linee elettroniche \textit{	\underline{interne alla CPU}}. Nell'8086, il BUS dati interno comprende 16 linee separate, ciascuna delle quali contiene un 1 o uno 0. Nei processori odierni si hanno fino a 64/128 linee separate.
			\item \textbf{external BUS}: utilizzato quando la CPU necessita di comunicare con dei \textit{\underline{dispositivi esterni}}, come una stampante. L'external BUS collega il processore agli adattatori, alla tastiera, al mouse, al disco rigido e ad altri dispositivi.
				I processori odierni hanno BUS esterni con 64/128 bit.
		\end{itemize}
		
	\end{block}
	
\end{frame}


\subsection[Internal BUS ed external BUS]{Internal BUS ed external BUS}
\begin{frame}
	\frametitle{Internal BUS ed external BUS}
	  
	\begin{figure}[!htbp]
		\centering
		\includegraphics[width=1.0\linewidth]{images/6_bus/bus_internal_external.pdf}
		%\caption{}
%		\label{}
	\end{figure}
\end{frame}


\begin{frame}
	\frametitle{Internal BUS ed external BUS}
	  
	\begin{figure}[!htbp]
		\centering
		\includegraphics[width=0.715\linewidth]{images/6_bus/bus_size.pdf}
		\caption{per il computer, ogni lettera dell'alfabeto è una diversa combinazione di otto 1 e 0 (in una codifica ASCII ad esempio). Ad esempio, la lettera D è 01000100 e la lettera E è 01000101. La figura dimostra che aumentando le dimensioni del BUS aumentano notevolmente le prestazioni di un computer, in modo simile a come l'aumento del numero di corsie di un'autostrada ne riduce la congestione.}
%		\label{}
	\end{figure}
\end{frame}



\subsection[Master \& Slave]{Master \& Slave}
\begin{frame}
	\frametitle{Master \& Slave}
	
	\begin{block}{}
		Nei sistemi informatici, l'architettura del bus può coinvolgere dispositivi che agiscono come "master" e "slave" durante le operazioni di comunicazione.
		\begin{itemize}
			\item Il \textbf{master} è il dispositivo che inizia e controlla l'operazione di trasferimento dati. Detiene il controllo del flusso di dati e inizia le richieste di lettura o scrittura ad altri dispositivi. Ad esempio, una CPU può essere il master quando richiede dati dalla memoria RAM.
			\item Lo \textbf{slave}, d'altro canto, risponde alle richieste del master e fornisce i dati richiesti o accetta i dati da scrivere. Un esempio tipico è un dispositivo di archiviazione come un'unità SSD o un'unità disco rigido, che risponde alle richieste di lettura o scrittura provenienti dalla CPU (master).
		\end{itemize}
		Questo concetto è fondamentale per garantire che le comunicazioni avvengano in modo coordinato e senza conflitti, consentendo al sistema di funzionare in modo efficiente e affidabile.
		
	\end{block}
	
\end{frame}



