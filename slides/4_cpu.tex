\section[La CPU]{La CPU}
\label{sec:cpu}
\sectionframe{images/covers/cover_cpu.jpeg}{La CPU}	 


\subsection[Central Processing Unit]{Central Processing Unit}
\begin{frame}
	\frametitle{Central Processing Unit}
	
%	\begin{block}{Central Processing Unit}
		Una CPU, \textbf{central processing unit} (unità centrale di elaborazione o processore centrale), indica un'unità o sottosistema logico e fisico che sovraintende alle funzionalità logiche di elaborazione principali di un computer.
		La CPU è un'elaborata combinazione di transistor che può essere definita \textit{circuito integrato}.\\~\\
		\pause
		All'interno della CPU individuiamo tre elementi fondamentali:
		\begin{itemize}
			\item \textbf{la CU}, \textit{Control Unit} (l’unità di controllo):\\
			coordina l'esecuzione delle operazioni da parte del processore
			\item \textbf{la ALU}, \textit{arithmetic-logic unit} (l’unità aritmetico-logica):\\
			si occupa di eseguire le operazioni aritmetico-logiche
			\item \textbf{i registri di memoria}:\\
			diverse \textit{celle di memoria} dedicate a scopi specifici che vengono utilizzati per il controllo dell'esecuzione di un programma.
		\end{itemize}
%	\end{block}
	
\end{frame}
