\section[I Sistemi]{I Sistemi}
\sectionframe{images/covers/cover_intro.jpg}{Intro}

\subsection[Definizioni: sistema e stato di un sistema]{Definizioni: sistema e stato di un sistema}

% Pastorello: intro_statistical_learning
% https://www.javatpoint.com/machine-learning
\begin{frame}
	\frametitle{Definizione di sistema}
	
	\begin{block}{Sistema}
		Un \textbf{sistema} è un insieme di elementi in relazione tra di loro secondo leggi ben precise che concorrono al raggiungimento di un obiettivo comune.
	\end{block}
	\begin{block}{Stato}
		Il sistema, in base al dati ricevuti in ingresso ed alla sua natura, evolve istante dopo istante in una nuova situazione che è chiamata \textbf{stato}.
	\end{block}
\end{frame}


\begin{frame}
	\frametitle{Definizione di stato di un sistema}
	
	\begin{block}{Lo stato}
		Lo \textbf{stato} di un sistema è l'insieme di informazioni necessarie e sufficienti per descrivere le condizioni in cui si trova un sistema in un qualunque istante.\\
		\vspace{0.5em}
		Ad esempio:
		\begin{itemize}
			\item Nel sistema \textbf{automobile}:\\
				stato = velocità alla quale si trova in un dato istante (0 se è ferma).
			\item Nel sistema \textbf{aeroplano}:\\
				stato = litri di carburante nel serbatoio, il numero di giri del motore, Il livello dell’ollo ecc.
		\end{itemize}
		\vspace{0.5em}
		Leggere i valori dello stato di un sistema in un determinato momento equivale a farne una fotografia istantanea.\\
%		I valori degli ingressi, uscite e stati in un sistema, vengono memorizzati in apposite variabili che prendono il nome di variabili d'ingresso, d'uscita e di stato.
		
	\end{block}
\end{frame}


\begin{frame}
	\frametitle{Esempi di sistemi e stati}
	
	\begin{block}{Esempi di sistemi e stati}
		Potremmo paragonare lo stato di un sistema ad una fotografia del sistema in un dato istante:
		\begin{itemize}
			\item \textbf{Sistema contatore del gas}: in tale contesto lo \textbf{stato} del sistema corrisponde alla cifra dei Smc (standard metro cubo) consumati fino a quell'istante.
			\item \textbf{Sistema distributore automatico delle bibite}: in tale contesto lo \textbf{stato} del sistema corrisponde alle bibite presenti nei vari slot ed al denaro per restituire il resto presente nella macchina.
		\end{itemize}
	\end{block}
\end{frame}



\subsection[Sistemi naturali, artificiali e misti]{Sistemi naturali, artificiali e misti}
\begin{frame}
	\frametitle{Sistemi naturali, artificiali e misti}
	
	\begin{block}{È possibile catalogare i sistemi in tre grandi categorie:}
		\begin{itemize}
			\item \textbf{Sistemi naturali}: \textit{sono i sistemi esistenti in natura} .\\
			Si cita come esempio il sistema solare. Anche per il corpo umano si utilizza il termine sistema inteso come insieme di componenti che, aggregati in modo opportuno, svolgono specifiche funzioni; si hanno per esempio il sistema digerente e il sistema circolatorio.
			\item \textbf{Sistemi artificiali}: \textit{sono i sistemi creati dall'uomo}.\\
			Esempi di sistemi artificiali sono il sistema computer, un distributore di bibite, un'ascensore oppure il sistema di regolazione della temperatura dell'acqua di una piscina.
			\item \textbf{Sistemi misti}: \textit{sono sistemi naturali sui quali l’uomo è intervenuto}.\\
			Si cita come esempio il caso di un sistema destinato alla produzione di energia elettrica che viene realizzato deviando un corso d'acqua.
		\end{itemize}
	\end{block}
\end{frame}


\subsection[Inputs, states e outputs]{Inputs, states e outputs}

\begin{frame}
	\frametitle{Inputs, states e outputs}
	
	\begin{block}{Potremmo rappresentare qualunque sistema X nel seguente modo:}
		\begin{itemize}
			\item Riceve un insieme di informazioni di partenza \textbf{I} che chiameremo \textbf{inputs}.
			\item Può modificare lo/gli stato/stati \textbf{S} nel quale si può trovare in un determinato istante.
			\item Fornisce un insieme di risposte \textbf{O} che chiameremo \textbf{outputs}.
		\end{itemize}
		
		\begin{figure}[!htbp]
			\centering
			\includegraphics[width=0.55\linewidth]{images/1_i_sistemi/sistemaX.pdf}
					%\caption{}
		\end{figure}
		\vspace{0.4em}
	\end{block}
\end{frame}


\begin{frame}
	\frametitle{Inputs, states e outputs}
	
	\begin{block}{Le grandezze che determinano un sistema:}
		Un sistema è basato su quattro grandezze principali\\
		(in generale indicheremo in maiuscolo l’insieme di tutte le variabili e in minuscolo la singola variabile):\vspace{1em}
		\begin{itemize}
			\item Ingressi ($\pmb{I}$): formati dal set delle variabili di ingresso ($\pmb{i_1}$, $\pmb{i_2}$, ..., $\pmb{i_n}$)
			\item Uscite ($\pmb{O}$): formate dal set delle variabili di uscita ($\pmb{o_1}$, $\pmb{o_2}$, ..., $\pmb{o_n}$)
			\item Stati ($\pmb{S}$): formati dal set delle variabili di stato ($\pmb{s_1}$, $\pmb{s_2}$, ..., $\pmb{s_n}$)
			\item Tempo ($\pmb{T}$): formati dalle variabili tempo ($\pmb{t_1}$, $\pmb{t_2}$, ..., $\pmb{t_n}$)\\nelle quali si studia il sistema
		\end{itemize}
	\end{block}
\end{frame}


\begin{frame}
	\frametitle{Inputs, states e outputs}
	
	\begin{block}{Le grandezze che determinano un sistema:}
		\vspace{1.5em}
		\begin{figure}[!htbp]
			\centering
			\includegraphics[width=0.65\linewidth]{images/1_i_sistemi/sistemaX2.pdf}
					%\caption{}
		\end{figure}
		\vspace{1.5em}
	\end{block}
\end{frame}



\begin{frame}
	\frametitle{Un esempio di sistema:}
	
%	\begin{block}{}
		\begin{itemize}
			\item \textbf{Set degli input}:
				\begin{itemize}
					\item[--] le monete inserite ($i_1$ = coins).
					\item[--] la scelta/selezione fatta sul distributore ($i_2$ = choice).
				\end{itemize}
			\item \textbf{Set degli stati}:
				\begin{itemize}
					\item[--] le monete all'interno del distributore ($s_1$ = coins).
					\item[--] le bibite presenti all'interno del distributore ($s_2$ = drinks).
				\end{itemize}
			\item \textbf{Set degli output}:
				\begin{itemize}
					\item[--] la bibita in uscita ($o_1$ = drink).
					\item[--] il resto ($s_2$ = change).
				\end{itemize}
		\end{itemize}
		\vspace{1.5em}
		\begin{figure}[!htbp]
			\centering
			\includegraphics[width=0.60\linewidth]{images/1_i_sistemi/sistemaX3.pdf}
					%\caption{}
		\end{figure}
		\vspace{1.5em}
%	\end{block}
\end{frame}




\subsection[Funzione di transizione di stato $f$]{Funzione di transizione di stato $f$}


\begin{frame}
%	\frametitle{La funzione di transizione dello stato}
	
	\begin{block}{La \textbf{funzione di transizione dello stato} ($\pmb{f}$)}
		è la funzione che determina quale sia il valore dello stato del sistema $\pmb{s}$ in un generico istante $\pmb{t}$.
		$$\pmb{s(t) = f\Big(s(t_0), i(t)\Big)}$$
		
		\begin{itemize}
			\item $\pmb{f}$ è la funzione di transizione dello stato
			\item $\pmb{s(t_0)}$ è lo stato iniziale del sistema al tempo $t_0$
			\item $\pmb{i(t)}$ sono tutti gli ingressi applicati al sistema dall'istante iniziale $t_0$ all'istante $t$
			\item $\pmb{s(t)}$ è quindi lo stato del sistema al tempo $\pmb{t}$.	\\
			Esso può essere ottenuto tramite la funzione di transizione dello stato $\pmb{f}$ conoscendo $\pmb{s(t_0)}$ e $\pmb{i(t)}$.
		\end{itemize}
		
		\begin{figure}[!htbp]
			\centering
			\includegraphics[width=0.50\linewidth]{images/1_i_sistemi/sistemaF.pdf}
					%\caption{}
		\end{figure}
	\end{block}
\end{frame}



\subsection[Funzione di trasformazione delle uscite $g$]{Funzione di trasformazione delle uscite}


\begin{frame}
%	\frametitle{Funzione di trasformazione delle uscite}
	
	\begin{block}{La \textbf{funzione di trasformazione delle uscite} ($\pmb{g}$)}
		è la funzione che determina quale valore avrà l'uscita $\pmb{o(t)}$ ad un generico istante $\pmb{t}$.
		$$\pmb{o(t) = g\Big(s(t), i(t)\Big)}$$
		
		\begin{itemize}
			\item $\pmb{g}$ è la funzione di trasformazione delle uscite
			\item $\pmb{s(t)}$ è lo stato del sistema al tempo $t$
			\item $\pmb{i(t)}$ sono tutti gli ingressi applicati al sistema dall'istante iniziale $t_0$ all'istante $t$
			\item $\pmb{o(t)}$ è quindi il valore che avrà l'uscita $\pmb{o(t)}$ ad un generico istante $\pmb{t}$, conoscendo il valore dello stato e degli ingressi nel medesimo istante.
		\end{itemize}
		
		\begin{figure}[!htbp]
			\centering
			\includegraphics[width=0.50\linewidth]{images/1_i_sistemi/sistemaF.pdf}
					%\caption{}
		\end{figure}
	\end{block}
\end{frame}



\subsection[Un esempio completo]{Un esempio completo}

\begin{frame}
	\frametitle{Esempio di un sistema di illuminazione}
	\begin{block}{Sistema di illuminazione:}
	
	Prendiamo in esame un semplice sistema di illuminazione come mostrato nello schema a fianco.
	
	
	\begin{columns}			
		\column{0.65\linewidth}
		\hspace{0.1em}
		Ci sono solo quattro stati ammessi: 
		\begin{itemize}
			\item buio, ovvero tutte le luci spente ($Dark$)
			\item solo la luce nell'ingresso accesa ($L_I$)
			\item solo la luce nel corridoio a sx accesa ($L_{LX}$)
			\item solo la luce nel corridoio a dx accesa ($L_{RX}$)
		\end{itemize}
		
		Il sistema riceve gli input esclusivamente attraverso i due bottoni $B_1$ e $B_2$, possono esser premuti solo uno per volta.
		Se non mantenuti premuti rimangono sollevati (entrambi su OFF).
					
		\column{0.35\linewidth}
		\begin{figure}[!htbp]
			\centering 
			\includegraphics[width=0.8\linewidth]{images/1_i_sistemi/sistemaLight.pdf}
					%\caption{}
		\end{figure}
		
	\end{columns}
	\end{block}
\end{frame}


\begin{frame}
	\frametitle{Esempio di un sistema di illuminazione}
	\begin{block}{Sistema di illuminazione, funzione di transizione dello stato:}
	
	Prendiamo in esame la funzione di transizione dello stato del semplice sistema di illuminazione.
	
	
	\begin{columns}			
		\column{0.6\linewidth}
\begin{small}
	
\begin{table}[]
\begin{tabular}{|
>{\columncolor[HTML]{C0C0C0}}c |c|c|c|c|}
\hline
\cellcolor[HTML]{FFFFFF}$\pmb{s}$\textbackslash $\pmb{i}$ & \cellcolor[HTML]{C0C0C0}\begin{tabular}[c]{@{}c@{}}$B_1$ OFF\\ $B_2$ OFF\end{tabular} & \cellcolor[HTML]{C0C0C0}\begin{tabular}[c]{@{}c@{}}$B_1$ ON\\ $B_2$ OFF\end{tabular} & \cellcolor[HTML]{C0C0C0}\begin{tabular}[c]{@{}c@{}}$B_1$ OFF\\ $B_2$ ON\end{tabular} & \cellcolor[HTML]{C0C0C0}\begin{tabular}[c]{@{}c@{}}$B_1$ ON\\ $B_2$ ON\end{tabular} \\ \hline
$Dark$ & $Dark$ & $L_I$ & $L_I$ & - \\ \hline
$L_I$ & $L_I$ & $L_{LX}$ & $L_{RX}$ & - \\ \hline
$L_{LX}$ & $L_{LX}$ & $Dark$ & $Dark$ & - \\ \hline
$L_{RX}$ & $L_{RX}$ & $Dark$ & $Dark$ & - \\ \hline                                                                        
\end{tabular}
\end{table}
\end{small}
					
		\column{0.4\linewidth}
		\begin{figure}[!htbp]
			\centering
			\includegraphics[width=0.80\linewidth]{images/1_i_sistemi/sistemaLight.pdf}
					%\caption{}
		\end{figure}
	\end{columns}
	\end{block}
\end{frame}




\begin{frame}
	\frametitle{Esempio di un sistema di illuminazione}
	\begin{block}{Sistema di illuminazione, funzione di transizione dello stato:}
	
	Prendiamo in esame la funzione di transizione dello stato del semplice sistema di illuminazione.
	
	
	\begin{columns}			
		\column{0.6\linewidth}
\begin{small}
	
\begin{table}[]
\begin{tabular}{|
>{\columncolor[HTML]{C0C0C0}}c |c|c|c|c|}
\hline
\cellcolor[HTML]{FFFFFF}$\pmb{s}$\textbackslash $\pmb{i}$ & \cellcolor[HTML]{C0C0C0}\begin{tabular}[c]{@{}c@{}}$B_1$ OFF\\ $B_2$ OFF\end{tabular} & \cellcolor[HTML]{C0C0C0}\begin{tabular}[c]{@{}c@{}}$B_1$ ON\\ $B_2$ OFF\end{tabular} & \cellcolor[HTML]{C0C0C0}\begin{tabular}[c]{@{}c@{}}$B_1$ OFF\\ $B_2$ ON\end{tabular} & \cellcolor[HTML]{C0C0C0}\begin{tabular}[c]{@{}c@{}}$B_1$ ON\\ $B_2$ ON\end{tabular} \\ \hline
$Dark$ & $Dark$ & $L_I$ & $L_I$ & - \\ \hline
$L_I$ & $L_I$ & $L_{LX}$ & $L_{RX}$ & - \\ \hline
$L_{LX}$ & $L_{LX}$ & $Dark$ & $Dark$ & - \\ \hline
$L_{RX}$ & $L_{RX}$ & $Dark$ & $Dark$ & - \\ \hline                                                                        
\end{tabular}
\end{table}
\end{small}
					
		\column{0.4\linewidth}
		\begin{figure}[!htbp]
			\centering
			\includegraphics[width=0.9\linewidth]{images/1_i_sistemi/sistemaLightF.pdf}
					%\caption{}
		\end{figure}
	\end{columns}
	\end{block}
\end{frame}



\begin{frame}
	\frametitle{Esempio di un sistema di illuminazione}
	\begin{block}{Sistema di illuminazione, funzione di trasformazione delle uscite:}
	
	La funzione di trasformazione delle uscite in questo specifico caso è funzione solo dello stato in cui il sistema si va a trovare, ovvero non tiene di conto dell'ultimo input ricevuto al tempo $t$ ma solo dello stato:
	
	\begin{itemize}
		\item $s=Dark \quad\rightarrow$ $output =$ luci spente
		\item $s=L_{I} \quad\:\quad\rightarrow$ $output =$ luce dell'ingresso accesa
		\item $s=L_{LX} \quad\;\;\rightarrow$ $output =$ luce del corridoio sx accesa
		\item $s=L_{RX} \quad\;\;\rightarrow$ $output =$ luce del corridoio sx accesa
	\end{itemize}
	
	\end{block}
\end{frame}


 
\subsection[Classificazione dei sistemi]{Classificazione dei sistemi}

\begin{frame}
	\frametitle{Classificazione dei sistemi}
	\begin{block}{Classificazione dei sistemi:}
		L'approccio più semplice per studiare e classificare i sistemi è chiamato \textbf{modello black box} (a scatola nera).\\
		In questo modello ci si limita ad analizzare il comportamento osservabile all'esterno di un sistema ignorandone la struttura ed il funzionamento interno.
		\begin{figure}[!htbp]
			\centering
			\includegraphics[width=0.8\linewidth]{images/1_i_sistemi/blackbox.pdf}
					%\caption{}
		\end{figure}
		Si esaminano quindi solo:
		\begin{itemize}
			\item gli elementi in ingresso del sistema (inputs) 
			\item gli elementi in uscita del sistema (outputs)
		\end{itemize}
%		dal sistema stesso.
	\end{block}
\end{frame}



\begin{frame}
	\frametitle{Classificazione sistemica}
	\begin{block}{Classificazione sistemica}
		Sulla base di tali osservazioni possiamo classificare un sistema identificando tipologie a due a due di natura opposta; si hanno pertanto sistemi:
		\begin{itemize}
			\item Variante o invariante
			\item Continuo o discreto
			\item Combinatorio o sequenziale
			\item Deterministico o stocastico
			\item Dinamico o statico
			\item Proprio o improprio
			\item ...
		\end{itemize}
		
		Lo stesso sistema può anche essere classificato in modi diversi, a seconda delle caratteristiche e dell'intervallo di tempo nel quale viene studiato.\\
%		Ad es. un sistema orologio analogico (con lancette) anche se analogico, può divenire discreto quando studiato in un tempo infinitesimale ($\approx 0$)
	\end{block}
\end{frame}


\subsubsection[Variante o invariante]{Variante o invariante}
\begin{frame}
	\frametitle{Variante o invariante}
	\begin{block}{Variante o invariante nel tempo}
		Un sistema è \textbf{invariante nel tempo} quando i parametri che lo caratterizzano rimangono invarianti nel tempo e quindi le leggi che legano le sollecitazioni alle risposte rimangono invariate nel tempo.\\~\\
		\textbf{ATTENZIONE}: non è variante un sistema quando variano gli ingressi e/o le uscite: in tal caso si dice che il sistema è dinamico.
	\end{block}
\end{frame}

\subsubsection[Continuo o discreto]{Continuo o discreto}
\begin{frame}
	\frametitle{Continuo o discreto}
	\begin{block}{Continuo o discreto}
		Un sistema può essere \textbf{discreto}:
		\begin{itemize}
			\item nell'\textbf{avanzamento}
			\item nelle \textbf{sollecitazioni}
			\item nelle \textbf{interazioni}
		\end{itemize}
		
		Un sistema si dice continuo se non è discreto in nessuno dei tre aspetti (avanzamento, sollecitazioni e interazioni).
	\end{block}
\end{frame}


\begin{frame}
	\frametitle{Continuo o discreto: nell'avanzamento}
	\begin{block}{Un sistema è discreto nell'avanzamento}
		quando non esistono tempi intermedi tra un istante $t_n$ e il suo successivo $t_{n+1}$ in cui viene studiato (ad es. nei sistemi di scansione temporale tramite un clock).\\
		Ad esempio: \textit{ripresa cinematografica microprocessore scandito dal clock}.
	\end{block}
	
	\begin{block}{Un sistema è discreto nelle sollecitazioni}
		quando l'insieme delle Variabili d'Ingresso (VI) è discreto.\\
		Ad esempio: \textit{un distributore di bibite} o \textit{un sistema di illuminazione}.
	\end{block}
	
	\begin{block}{Un sistema è discreto nelle interazioni}
		quando la funzione di transizione e/o la funzione di trasformazione sono entrambe discrete.\\
		Ad esempio: \textit{un orologio digitale con ingresso continuo ma uscita discreta}.
	\end{block}
\end{frame}


\begin{frame}
	\frametitle{Continuo o discreto}
	\begin{block}{Continuo o discreto}		
		Un tipico esempio di sistema continuo è rappresentato da:
		\begin{itemize}
			\item un albero che produce frutti, 
			\item una mucca che produce latte, 
			\item ecc...
		\end{itemize}
		
	\end{block}
\end{frame}

\subsubsection[Combinatorio o sequenziale]{Combinatorio o sequenziale}
\begin{frame}
	\frametitle{Combinatorio o sequenziale}
	\begin{block}{Combinatorio o sequenziale}
		abc
	\end{block}
\end{frame}

\subsubsection[Deterministico o stocastico]{Deterministico o stocastico}
\begin{frame}
	\frametitle{Deterministico o stocastico}
	\begin{block}{Deterministico o stocastico}
		abc
	\end{block}
\end{frame}

\subsubsection[Dinamico o statico]{Dinamico o statico}
\begin{frame}
	\frametitle{Dinamico o statico}
	\begin{block}{Dinamico o statico}
		abc
	\end{block}
\end{frame}

\subsubsection[Proprio o improprio]{Proprio o improprio}
\begin{frame}
	\frametitle{Proprio o improprio}
	\begin{block}{Proprio o improprio}
		abc
	\end{block}
\end{frame}
